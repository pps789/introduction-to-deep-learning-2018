\documentclass{article}
\usepackage{bm}
\usepackage{amsmath}
\usepackage{graphicx}
\usepackage{mdwlist}
\usepackage[colorlinks=true]{hyperref}
\usepackage{geometry}
\geometry{margin=1in}
\geometry{headheight=2in}
\geometry{top=2in}
\usepackage{palatino}
%\renewcommand{\rmdefault}{palatino}
\usepackage{fancyhdr}
%\pagestyle{fancy}
\rhead{}
\lhead{}
\chead{%
  {\vbox{%
      \vspace{2mm}
      \large
      Introduction to Deep Learning M2177.0043 \hfill
\\
      Seoul National University
      \\[4mm]
      Homework \#(\textbf{1})\\
      \textbf{Sanghyeok Park}
    }
  }
}


\usepackage{paralist}

\usepackage{todonotes}
\setlength{\marginparwidth}{2.15cm}

\usepackage{tikz}
\usetikzlibrary{positioning,shapes,backgrounds}

\begin{document}
\pagestyle{fancy}

\section*{INSTRUCTIONS}

\begin{itemize*}
\item Anything
  that is received after the deadline will be considered to be late and we do not receive late homeworks. We do however ignore your lowest homework grade. 
\item Answers to every theory questions need to be submitted
  electronically on ETL. Only PDF generated from LaTex is accepted.
\item Make sure you prepare the answers to each question
  separately. This helps us dispatch the problems to different graders.
\item Collaboration on solving the homework is allowed. Discussions
  are encouraged but you should think about the problems on your own. 
\item If you do collaborate with someone or use a book or website, you
  are expected to write up your solution independently.  That is,
  close the book and all of your notes before starting to write up
  your solution. 
\end{itemize*}

%!TEX root = hw1.tex

%% Q1
\section{Q1}

\subsection{}
\begin{center}
	\begin{tabular}{| c | c |}
		\hline
        Sanghyeok & Park \\ \hline
        Computer Science \& Engineering & 2013-11400 \\
		\hline
	\end{tabular}
\end{center}

\subsection{}
\begin{center}
    \includegraphics{./profile.jpg}
\end{center}

%% Q2
\section{Q2}

\subsection{}
Let positions of each chamber $1, 2, \cdots, i, \cdots, 6$,
and let's assume $pos=i$ means the position of bullet is $i$.
Let's assume the probability of all possibility is equal, i.e. $P(\{i\})=\frac{1}{6}$ for $i=1, \cdots, 6$.
First player will not survive in case of $pos=1, pos=3, pos=5$.
Otherwise, second player will not survive.
First player's survival probability is $P(\{2,4,6\})=\frac{1}{2}$.
Second player's survive probability is also $P(\{1,3,5\})=\frac{1}{2}$.
Therefore, I could choose any turn.

\subsection{}
If we spin the barrel after each trigger pull, then probability of death of each turn is $\frac{1}{6}$.
For convenience, denote X when player dies, and O when player survives.
First player will die in case of sequences: X, OOX, OOOOX, so on.
Each `X' has probability $\frac{1}{6}$, and `O' has $\frac{5}{6}$.
And we assume each trigger is independent.
Therefore, the death probability of first player is
$\frac{1}{6} + (\frac{5}{6})^{2}\cdot\frac{1}{6} + (\frac{5}{6})^{4} + \cdots
= \frac{\frac{1}{6}}{1 - (\frac{5}{6})^{2}} = \frac{6}{11}$.
And the death probability of second player is $1 - \frac{6}{11} = \frac{5}{11}$.
Therefore, I should go second.

\subsection{}
From now, I assume that problems wanted to maximize probabilities of survive in my first turn. \\

From now denote $pos=(i,j)$ which means the positions of bullet is $i$ and $j$.
To remove duplicated cases, we should restrict $1 \leq i < j \leq 6$.
Let's define $C=\{(i,j) | 1 \leq i < j \leq 6\}$.
There are total $15$ cases, so $P(\{(1,2)\}) = \cdots = P(\{(5,6)\}) = \frac{1}{15}$.
If I spin the barrel, I will die if and only if $i=1$ where $pos=(i,j)$.
Therefore, probability of death is
$P(\{(i,j) \in C | i=1\})
= P(\{(1,2), \cdots, (1,6)\})
= \frac{5}{15} = \frac{1}{3}$. \\

In case of non-spinning, let $A=\{(i,j) | 2 \leq i < j \leq 6\}$.
Set $A$ means the cases that first player not died
    because no bullet is in the position $1$ chamber.
Now, let $B=\{(i,j) \in C | i=2\}$.
Set $B$ means the cases that second player will die
    because the first bullet is in the chamber of second position.
We should calculate $P(B|A)$.
$|A|=4+3+2+1=10$, so $P(A) = \frac{10}{15} = \frac{2}{3}$.
$|B|=4$, so $P(B) = \frac{4}{15}$.
Since $B \subseteq A$, $P(A \cap B) = P(B) = \frac{4}{15}$.
Therefore, $P(B|A) = \frac{P(A \cap B)}{P(A)} = \frac{2}{5}$,
    it is larger than spinning case. \\

Therefore, I should spin the barrel.

\subsection{}
Just do same calcuation in Q2.3, but there are only total 6 cases,
$C = \{(1,2), (2,3), \cdots, (5,6), (1,6)\}$.
So $P(\{(1,2)\}) = \cdots = P(\{(5,6)\}) = \frac{1}{6}$. \\

If I spin the barrel, probability of death is
$P(\{(i,j) \in C | i=1\})
= P(\{(1,2),(1,6)\})
= \frac{2}{6} = \frac{1}{3}$. \\

In case of non-spinning, let $A=\{(i,j) \in C | 2 \leq i\}$.
Set $A$ means the cases that first player not died
    because no bullet is in the position $1$ chamber.
Now, let $B=\{(i,j) \in C | i=2\}$.
Set $B$ means the cases that second player will die
    because the first bullet is in the chamber of even position.
We should calculate $P(B|A)$.
$A=\{(2,3),(3,4),(4,5),(5,6)\}$, so $P(A)=\frac{4}{6}$.
$B=\{(2,3)\}$, so $P(B)=\frac{1}{6}$.
Since $B \subseteq A$, $P(A \cap B) = P(B) = \frac{1}{6}$.
Therefore, $P(B|A) = \frac{P(A \cap B)}{P(A)} = \frac{1}{4}$,
    it is smaller than spinning case. \\

Therefore, I should not spin the barrel.

%% Q3
\section{Q3}

\subsection{}
At first, $0 \leq F_X(x) \leq 1$ since $F_X$ is cdf itself.
Therefore, $Y$ is distributed in $[0,1]$.
Let's consider cdf of $Y$.
$F_Y(y) = P(F_X(X) \leq y)$ for $0 < y < 1$, $F_Y(y) = 0$ for $y \leq 0$,
    and $F_Y(y) = 1$ for $y \geq 1$.
In case of $0 < y < 1$, there exists a real number $x$ such that $F_X(x)=y$
    since $F_X$ is continuous.
Since $F_X$ is monotonic increasing and continuous,
    there exists a real number $x_0$
    such that $x_0 = \sup\limits_{x}\{F_X(x)=y\}$, $F_X(x_0)=y$.
Therefore,
$P(F_X(X) \leq y)
= P(F_X(X) \leq F_X(x_0))
= P(X \leq x_0)
= F_X(x_0)
= y$.
Cdf of uniform distribution $U$ in $[0,1]$ is also $F_U(u)=u$ for $0 < u < 1$.
Since pdf of $Y$ is continuous and its cdf is same of $U$'s cdf in $0 < y < 1$,
    pdf of $Y$ and $U$ is same in $0 < y < 1$, i.e. $f_Y(y)=f_U(y)$ for $0 < y < 1$.
Therefore, $Y$ is unofirmly distributed in $[0,1]$.

\subsection{}
Before start, let's restrict domain of x where $\{x|f_X(x) > 0\}$ (I found some hints in eTL).
Let's denote such domain $A$.
Then, we can assume that $F_X^{-1}$ exists.
To find cdf of $F_X^{-1}$, we should calculate $P(F_X^{-1}(U) \leq x)$ for $x \in A$.
Since we restrict the domain of x, there exists $u \in (0,1)$ s.t. $F_X^{-1}(u)=x$.
Since $F_X$ is continuous and strictly increasing (in such domain),
    $F_X^{-1}(U) \leq x \iff u \leq F_X(x)$.
Therefore, cdf of $F_X^{-1}$ at $x$ is
$P(F_X^{-1}(U) \leq x)
= P(U \leq F_X(x) = u)
= F_U(u)
= u
= F_X(x)$.

%% Q4
\section{Q4}
Let shares of stock $B$ will remain $k$.
Let RV $X$: the price of stock $A$, and $Y$: the price of stock $B$.
The variance of total stocks is
$V(X+kY) = V(X) + k^2V(Y) + 2kCov(X,Y)
= \sigma_A^2 + k^2\sigma_B^2 + 2k\sigma_A\sigma_B\rho$.
This quadratic equation is minimized when
$k=-\frac{(\text{coefficient of }k)}{2(\text{coefficient of }k^2)}
=-\frac{2\sigma_A\sigma_B\rho}{2\sigma_B^2}
=-\rho\frac{\sigma_A}{\sigma_B}$.
Therefore, if $\rho \geq 0$ then sell all shares of $B$.
If $\rho < 0$ then sell shraes of $B$ until $-\rho\frac{\sigma_A}{\sigma_B}$ remains.

%% Q5
\section{Q5}

\subsection{}
To maximize variance, we should scatter data into such range $[a,b]$.
I thought that the largest variance case is, the data appears only at $a$ and $b$ (`both ends' of range).
So I defined pmf of $X$ s.t. $P(X=a)=P(x=b)=\frac{1}{2}$.
In this case,
$V(X) = E(X^2) - E(X)^2
= ((a^2 + b^2)\frac{1}{2} - ((a+b)\frac{1}{2})^2)
= (2a^2+2b^2-(a^2+2ab+b^2))\frac{1}{4}
= \frac{a^2-2ab+b^2}{4}
= \frac{(a-b)^2}{4}$. \\

However, to show this value is maximum was very hard.
I found this link, \href{https://stats.stackexchange.com/questions/45588/variance-of-a-bounded-random-variable}{here}.
I got big hint from above link, reparameterize and $E[X^2] \leq E[X]$ if $X$ takes value only in $[0,1]$.
To prove $V(X) \leq \frac{(a-b)^2}{4}$, take any random variable $X$ s.t. it takes value only in range $[a,b]$.
If $a=b$, there is nothing to show.
In case of $a \neq b$, Let $Y=\frac{X-a}{b-a}$.
$V(Y)=\frac{V(X)}{(b-a)^2}$, so it is sufficient to show that $V(Y) \leq \frac{1}{4}$.
$0 \leq Y \leq 1$ implies $Y^2 \leq Y$.
Therefore, $E[Y^2] \leq E[Y]$.
Let $E[Y]=u$, then
$V(Y) = E[Y^2] - u^2
\leq u - u^2
= -(u-\frac{1}{2})^2 + \frac{1}{4}
\leq \frac{1}{4}$.
Equality holds when $u=E[Y]=\frac{1}{2}$.

\subsection{}
Already solved in above question; define pmf of $X$ s.t. $P(X=a)=P(x=b)=\frac{1}{2}$.
Otherwise, $P(X=t)=0 (t \neq a, t \neq b)$.
I already show that $V(X) = \frac{(a-b)^2}{4}$.

\end{document}
